\documentclass{endm}
\usepackage{endmmacro}

\usepackage{amssymb}
\usepackage{amsmath}
\usepackage{graphicx}
\usepackage[retainorgcmds]{IEEEtrantools}
\usepackage{rotating}
\usepackage{booktabs}
\usepackage{siunitx}
\usepackage{pdflscape}
% disable labelindent
\let\labelindent\relax
\usepackage{enumitem}
\usepackage{multirow}
\usepackage[normalem]{ulem}
\useunder{\uline}{\ul}{}
\usepackage{setspace} % \doublespacing
\usepackage[utf8]{inputenc}
% \usepackage[numbers,square]{natbib}
% \usepackage{algorithm, algorithmic}
\usepackage{rotating}
\usepackage{subfigure}

% tikz libraries
\usepackage{tkz-graph}
\usepackage{verbatim}
\usetikzlibrary{arrows,arrows.meta,shapes,decorations.pathmorphing, decorations.markings}
\let\svtikzpicture\tikzpicture
\def\tikzpicture{\noindent\svtikzpicture}

\def\lastname{Carvalho, Simpson}

\begin{document}

% DO NOT REMOVE: Creates space for Elsevier logo, ScienceDirect logo
% and ENDM logo
\begin{verbatim}\end{verbatim}\vspace{2.5cm}

\begin{frontmatter}

\title{Article title}

\author{Iago A. Carvalho\thanksref{all}\thanksref{ufmg}}, and
\author{Homer J. Simpson\thanksref{fox}}

% \address[ufmg]{Computer Science Department\\ Universidade Federal de Minas Gerais\\ Belo Horizonte, Brazil.}
% \address[ubp]{{Laboratoire LIMOS, CNRS-UMR 6158\\ Universit\'{e} Clermont Auvergne\\ Clermont-Ferrand, France.}}

\thanks[all]{This study was financed in part by the \emph{Coordenação de Aperfeiçoamento de Pessoal de Nível Superior - Brasil} (CAPES) - Finance Code 001, the \emph{Conselho Nacional de Desenvolvimento Científico e Tecnológico - Brasil} (CNPq), and the \emph{Fundação de Amparo à Pesquisa de Minas Gerais - Brasil} (FAPEMIG).}
\thanks[ufmg]{Department of Computer Science, Universidade Federal de Minas Gerais, Av. Antônio Carlos 6627, Belo Horizonte, MG 31270-010, Brazil \\Email: \texttt{\normalshape iagoac@dcc.ufmg.br}}
\thanks[fox]{Twenty Century Fox, Springfield, USA \\ Email:
\texttt{\normalshape homer@thesimpsons.com}}

\begin{abstract}
There will be an abstract here.
\end{abstract}

\begin{keyword}
Keyword 1, keyword 2, keyword 3, keyword 4, keyword 5
\end{keyword}

\end{frontmatter}

\section{Introduction}\label{sec:intro}

\bibliographystyle{endm}
\bibliography{bibsample}

\end{document}
